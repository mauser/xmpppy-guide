\section{Receiving messages}

Receiving messages is a little bit more complicated than sending messages. You need an event loop and an handler to do this.
\newline
Message handlers are a basic concept for acting on events. This means that you have to tell the xmpp.Client Object which method it should call if a message arrives. But first of all explanations, have a look at the code:
\newline
\newline
\begin{verbatim}
1  #!/usr/bin/python
2  import sys
3  import xmpp
4  import os
5  import signal
6  import time
7
8  def messageCB(conn,msg):
9      print "Sender: " + str(msg.getFrom())
10     print "Content: " + str(msg.getBody())
11     print msg
12
13
15 def StepOn(conn):
16    try:
17        conn.Process(1)
18    except KeyboardInterrupt:
19	    return 0
20    return 1
21
22 def GoOn(conn):
23    while StepOn(conn):
24	    pass
25
26
27 def main():
28
29
30	jid="mausers_jukebox@jabber.ccc.de"
31	pwd="music"
32
33	jid=xmpp.protocol.JID(jid)
34	cl = xmpp.Client(jid.getDomain(), debug=[])
35	cl.connect()
36	cl.auth(jid.getNode(),pwd)
37	cl.RegisterHandler('message', messageCB)
38	#cl.sendInitPresence()
39	GoOn(cl)
40
41 main()
\end{verbatim}
We should focus on the main() method to understand how everything works. The most code should appear familiar to you. Line 37 holds one new method: \begin{verbatim}	RegisterHandler('message', messageCB)\end{verbatim}
As you may have already guessed, the method "messageCB" is registered as a callback handler for incoming messages. So if you want to react an incoming messages, write a method called "messageCB" (as in line 8)
and place your message-handling code here. Be sure that your method takes 2 parameters.
The first parameter is the connection.The second parameter an x.protocol.Message instance. It is printed on your terminal if a message arrives. Search the API-Docs for it and experiment with the given methods gather experience. Maybe you could combine it with the example from Chapter 1 to react on certain messages.
\newline
\newline
If this example doesn't work for you, uncomment line 38. cl.sendInitPresence() tells our server that we're online. Some servers (for example Google's gtalk service) won't send messages to you if you're not marked as 'online'. A detailed look on presence handling is given in the next chapter.


\section{Setting the ressource}

The ressource of a jid is a nice jabber feature. With the use of different ressources, a jid can be logged into a server server several times. This means you can be online with more then one client 	at the same time.
Setting the jid is a very simple thing: A third argument has to be passed to the auth() method.
So the authentication step may look like this:



\begin{verbatim}
cl.auth(jid.getNode(),pwd,"laptop")
\end{verbatim}


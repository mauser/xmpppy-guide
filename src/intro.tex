%intro.tex
\section{Sending messages}

And now for something completely different.
Every tutorial on programming starts with a tiny "Hello World" program, so why break with this tradition.
Our program connects to a jabber server, authenticates itself with a username/password combination and sends a "Hello World!" message to a specified jid. It doesn't make much sense, but its a good start. Maybe you can use it  in shell scripts or something..

\begin{verbatim}
$$replace_with_send.py$$
\end{verbatim}
\newpage
\subsection{Connecting to the server}
The first thing to do in every jabber session is to connect to your jabber server.
Line 11 shows you how to do this. xmpp.Client() returns a Client instance, which is our basic object.
You can think of the client-object as your connection to the server. If your server uses an unusual port, you can pass it to the xmpp.Client constructor. Its signature is: xmpp.Client(server,port,debug).
\newline
\newline
The most work happens in Line 14 and 18. We're connecting to the server and trying to authenticate. That means the server just checks if our password is correct. jid.getNode() returns everything before the "@" in your jid.
\newline
\newline
You should check the return types of cl.connect() and cl.auth(). If you omit these checks strange errors could appear. If the cl.connect() fails (maybe because the jabber server is offline) it returns  an empty string. cl.auth() returns "None", for example if you provide a wrong password.
\newline
\newline
Line 23 finally does the magic: It sends a message (with content \textit{msg}) to \textit{recipient}.
After that, we should disconnect from the server to make a gracefull exit.

\subsection{Messages and message types}

You may have noticed that message you received from that script was shown to you on a different way than normal chat messages. If you use a pure jabber client like psi, which is very near to the jabber standard, the message might be shown like an email or something. That is because the xmpp protocol defines multiple message types. Less xmpp-compliant messengers like kopete make no difference between them.
\newline
\newline
RFC 3921 2.1.1 defines 5 message types:	chat,error,groupchat,headline and normal. For detailed informations see \textit{http://www.apps.ietf.org/rfc/rfc3921.html}.
\newline
At this time we want to focus on 'chat' and 'normal'. In our example above, we did not define any message type, so
psi interprets this as "normal". "normal" is described as a single message, without history. I suppose
you want to change that behaviour to 'chat' messages, so we have to set the type explicitly.
Now switch to the API and look after the methods of xmpp.Protocol. You'll discover a method called setType.
That sounds suitable, eh? Change line 23 to:
\begin{verbatim}
cl.send(xmpp.protocol.Message(recipient,msg,"chat"))
\end{verbatim}
















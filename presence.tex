%presence.tex
\section{Handling presence events}

\subsection{Simple presence handling}

Presence events are those messages which contain informations about your status and subscription.
You may have wondered why our bot from example 2 wasn't shown as "online" in your contact list.
This was because we didn't notify the server that the bot is online. This could be done by "sendInitPresence()". Go back to example 2 and uncomment the appropriate line. If the bot is in your roster, he will be marked as "online".
\newline
\newline
The next thing about presence covers subscription. "subscription" in generally means: "Allow somebody to see if you're online and allow him to add you to his roster".
To handle this events, we have to register a presence handler. This is done on the same way as the message handler in example 2.

\begin{verbatim}
1  #!/usr/bin/python
2  import sys
3  import xmpp
4  import os
5  import signal
6  import time
7
8  def presenceCB(conn,msg):
9	print str(msg)
10	prs_type=msg.getType()
11	who=msg.getFrom()
12	if prs_type == "subscribe":
13		conn.send(xmpp.Presence(to=who, typ = 'subscribed'))
14		conn.send(xmpp.Presence(to=who, typ = 'subscribe'))
15
16
17  def StepOn(conn):
18    try:
19        conn.Process(1)
20    except KeyboardInterrupt:
21	    return 0
22    return 1
23
24 def GoOn(conn):
25    while StepOn(conn):
26	    pass
27
28
29 def main():
30	jid="user@domain.tld"
31	pwd="secret"
32
33	jid=xmpp.protocol.JID(jid)
34
35	cl = xmpp.Client(jid.getDomain(), debug=[])
36
37	cl.connect()
38
39	cl.auth(jid.getNode(),pwd)
40
41
42	cl.RegisterHandler('presence', presenceCB)
43	cl.sendInitPresence()
44
45	GoOn(cl)
46
47 main()
\end{verbatim}
This is the most simple presence handler. When you receive a message containing "subscribe", return a "subscribed" answer and ask him for subscription. That means subscribing everyone who asks.
You can imagine that this is not the right thing for real world applications. I prefer limits like a maximal roster size and a policy to add users to the roster. This may differ for your bot..
\newline
Think about who should be able to use your bot and block everyone else.
\newpage
\subsection{Retrieving the status}


Many applications need the status of an user. They want to know if he's online or offline or maybe only away.
This is not as easy as it seems. There's no direct way to get the Status of a given jid. You have to analyse the presence messages the oponnents send. If a user gets online or if you go online, everyone who has subscribed to you will send a presence message. Of course this works only if the user is online.
If a user logs out, he sends you a presence message with the Status "Logged out". Everytime he changes his status, you will receive a corresponding status message.
My workaround for the problem: keep track of the actual status with a dictionary. Take the jid as the key
and set their value if you receive a presence message for the jid.
\newline
\newline
If you know a better way to do this, please contact me !


\section{Roster management}
The roster is the jabber synonym for the better known term "contact list". You can easily fetch, add or remove entries.
In the most cases you won't have to deal with this, because the most actions are done automatically.
For example if you subscribe to someones presence, he will be added to your roster.
A common example for a manual use of roster functions is to retrieve the jids contained in your roster.

Therefore xmpppy offers you a higher level class to operate on rosters: x.roster.Roster
\newline
The class should be self-explanatory if you have a look at
\newline \textit{http://xmpppy.sourceforge.net/apidocs/public/x.roster.Roster-class.html}.
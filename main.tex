\documentclass[a4paper,10pt]{report}


\title{xmpppy: a practical guide}

\author{Sebastian Moors \\ sebastian.moors@gmail.com}


\date{12.09.2006 \\[1cm] Version 0.02a}

\begin{document}

\begin{titlepage}
\maketitle


\end{titlepage}
\tableofcontents
\newpage




\chapter{Introduction}
\section{Motivation}
This tutorial was started because the author wrote a small bot and figured out that there was not enough documentation available for xmpppy. The small examples distributed with it are fine, but when you're building you're own bot and fiddling around with things like rostermanagement, you wish that there's something between the bare API-Specs and the small examples. So i decided to write down my experiences and publish them..
\newline
This paper should help you to get started and covers the basic issues of programming with xmpppy.
If you want to programm a bot, you will find some hints in the last chapters.
I assume that you have already python knowledge, python basics are not covered by this tutorial.
\newline
\newline
Furthermore no Jabber specific terms (such as "jid") will be explained here. Please use google to clarify this things.
\newline
\newline
This tutorial was written for a linux audience. I don't know if xmpppy works on Windows or *nix, so you've got to solve os-dependend problems by yourself.
\newline
\newline
If you like to contribute to this document or found a bug feel free to contact me.
\newline
\newline Jabber/Mail: \textit{sebastian.moors@gmail.com}





\section{What's so cool about jabber ?}
Jabber/xmpp is an Open Source instant messaging protocol. It is described in the RFCs
3920 - 3923 and is based on XML. A lot of Open Source and even commercial servers, clients and libraries are available.
It has several advanced features such as encryption and conferences.
Jabber beats ICQ and other protocols because the closed protocols are changing often and the libraries have to deal
with that fact. Informations about these protocols are often the result of reverse engineering. This work has to be
done every time the protocol changes. As you might know, this is a great problem for small software projects..
\newline
\\

\section{Why xmpppy ?}


I used xmpppy because it was available as a debian package and it's small. The design of xmpppy was exactly the thing i was looking for. There are high level functions for sending messages, but you can also build your messages as xml-strings if you wish to or if you need special features.
\newline
xmpppy is the inofficial successor of jabber.py, which i used before. The jabber.py project is dead now, so i migrated to xmpppy, which inherited some code from jabber.py and has a similar API.
\newline
If you're interested in other libraries, have a look at \textit{http://www.jabber.org}.


\section{Existing documentation}
Before you start with this tutorial you should gather informations about xmpppy and jabber.
You will need the API-Overview which is available at the project homepage. There are also some example programs available. The examples in this book are mainly based on these online examples.
\newline
\newline
Spend some time on the jabber.org or ask the wikipedia. If you're not new to this things, skip this..
\newline
\newline
For the best available documentation on jabber in general you should take a look at the RFCs. They're easier to read than you might expect and contain a lot of useful informations.
\newline
\newline
If you look for a concrete implementation of something, have a look at the projects using xmpppy under "Ressources".
\newline
\newline
If you want to get in contact with the xmpppy developers or just want to keep track of the current development, join the xmpppy-devel mailing list at \textit{https://lists.sourceforge.net/lists/listinfo/xmpppy-devel}.

\section{License}

This work is licensed under the creative commons license Attribution-NonCommercial-NoDerivs 2.0.
You can obtain it at \textit{http://creativecommons.org/licenses/by\-nc\-nd/2.0/deed.en\_GB}


%installation.tex
\chapter{Installation}



Use the package-management tool of your distribution to determine if a xmpppy package exists.
If there are no adequate packages or if you want the newest version, you have to build the software from the sources.
\newline
Get the newest tarball from the download page at the project's homepage (\textit{http://xmpppy.sourceforge.net}) and extract it into your home folder (\textit{The filenames and URLs are examples}
\begin{verbatim}
wget http://optusnet.dl.sourceforge.net/sourceforge/xmpppy/xmpppy-0.3.1.tar.gz
tar xzf xmpppy-0.3.1.tar.gz
cd xmpppy-0.3.1
python setup.py install
\end{verbatim}
If you need more informations, have a look at the README file distributed with xmpppy.



\chapter{Prerequisites}

Before we start with coding, you should check some preconditions. An important thing is the right jabber client. I would therefore recommend the use of psi. Psi is a tiny but feature-rich jabber client. It supports service browsing and it implements the most features as described in the RFC. Especially the group chat support is better implemented than in the most other clients.
\newline
Of course you can use a mutliprotocol client like Gaim or kopete too, but they implement some things on a different way.. For example gaim makes no difference between the multiple message types (chat, message..).
\newline
\newline
If you want to develop something with groupchat, you may want to setup your own jabber server.
This may be recommended if you're new to xmpppy and you don't know exactly what you're doing.
Think about the possibility of disturbing other people or - in the worst case - killing a server.
If your bot gets out of control and sends every 10ms a message to the server, the server admin might be a little bit disgusted of your public beta testing. So keep in mind that there are other people (and of course bots) out there while your bot uses a public server.
\newline
\newline
Furthermore i assume that you have already registered a jabber account.
\newline
\newline
If you're completely new to jabber, you should read something about the underlying techniques such as XML-Streams. A good starting point is the xmpp-core RFC.

\chapter{Basic concepts}

%intro.tex
\section{Sending messages}

And now for something completely different.
Every tutorial on programming starts with a tiny "Hello World" program, so why break with this tradition.
Our program connects to a jabber server, authenticates itself with a username/password combination and sends a "Hello World!" message to a specified jid. It doesn't make much sense, but its a good start. Maybe you can use it  in shell scripts or something..

\begin{verbatim}
$$replace_with_send.py$$
\end{verbatim}
\newpage
\subsection{Connecting to the server}
The first thing to do in every jabber session is to connect to your jabber server.
Line 11 shows you how to do this. xmpp.Client() returns a Client instance, which is our basic object.
You can think of the client-object as your connection to the server. If your server uses an unusual port, you can pass it to the xmpp.Client constructor. Its signature is: xmpp.Client(server,port,debug).
\newline
\newline
The most work happens in Line 14 and 18. We're connecting to the server and trying to authenticate. That means the server just checks if our password is correct. jid.getNode() returns everything before the "@" in your jid.
\newline
\newline
You should check the return types of cl.connect() and cl.auth(). If you omit these checks strange errors could appear. If the cl.connect() fails (maybe because the jabber server is offline) it returns  an empty string. cl.auth() returns "None", for example if you provide a wrong password.
\newline
\newline
Line 23 finally does the magic: It sends a message (with content \textit{msg}) to \textit{recipient}.
After that, we should disconnect from the server to make a gracefull exit.

\subsection{Messages and message types}

You may have noticed that message you received from that script was shown to you on a different way than normal chat messages. If you use a pure jabber client like psi, which is very near to the jabber standart, the message might be shown like an email or something. That is because the xmpp protocol defines multiple message types. Less xmpp-compliant messengers like kopete make no difference between them.
\newline
\newline
RFC 3921 2.1.1 defines 5 message types:	chat,error,groupchat,headline and normal. For detailed informations see \textit{http://www.apps.ietf.org/rfc/rfc3921.html}.
\newline
At this time we want to focus on 'chat' and 'normal'. In our example above, we did not define any message type, so
psi interprets this as "normal". "normal" is described as a single message, without history. I suppose
you want to change that behaviour to 'chat' messages, so we have to set the type explicitly.
Now switch to the API and look after the methods of xmpp.Protocol. You'll discover a method called setType.
That sounds suitable, eh? Change line 23 to:
\begin{verbatim}
cl.send(xmpp.protocol.Message(recipient,msg,"chat"))
\end{verbatim}
















\section{Receiving messages}

Receiving messages is a little bit more complicated than sending messages. You need an event loop and an handler to do this.
\newline
Message handlers are a basic concept for acting on events. This means that you have to tell the xmpp.Client Object which method it should call if a message arrives. But first of all explanations, have a look at the code:
\newline
\newline
\begin{verbatim}
1  #!/usr/bin/python
2  import sys
3  import xmpp
4  import os
5  import signal
6  import time
7
8  def messageCB(conn,msg):
9      print "Sender: " + str(msg.getFrom())
10     print "Content: " + str(msg.getBody())
11     print msg
12
13
15 def StepOn(conn):
16    try:
17        conn.Process(1)
18    except KeyboardInterrupt:
19	    return 0
20    return 1
21
22 def GoOn(conn):
23    while StepOn(conn):
24	    pass
25
26
27 def main():
28
29
30      jid="user@domain.tld"
31      pwd="secret"
32
33      jid=xmpp.protocol.JID(jid)
34      cl = xmpp.Client(jid.getDomain(), debug=[])
35      cl.connect()
36      cl.auth(jid.getNode(),pwd)
37      cl.RegisterHandler('message', messageCB)
38      #cl.sendInitPresence()
39      GoOn(cl)
40
41 main()
\end{verbatim}
We should focus on the main() method to understand how everything works. The most code should appear familiar to you. Line 37 holds one new method: \begin{verbatim}	RegisterHandler('message', messageCB)\end{verbatim}
As you may have already guessed, the method "messageCB" is registered as a callback handler for incoming messages. So if you want to react an incoming messages, write a method called "messageCB" (as in line 8)
and place your message-handling code here. Be sure that your method takes 2 parameters.
The first parameter is the connection.The second parameter an x.protocol.Message instance. It is printed on your terminal if a message arrives. Search the API-Docs for it and experiment with the given methods gather experience. Maybe you could combine it with the example from Chapter 1 to react on certain messages.
\newline
\newline
If this example doesn't work for you, uncomment line 38. cl.sendInitPresence() tells our server that we're online. Some servers (for example Google's gtalk service) won't send messages to you if you're not marked as 'online'. A detailed look on presence handling is given in the next chapter.


\section{Setting the resource}

The resource of a jid is a nice jabber feature. With the use of different resources, a jid can be logged into a server server several times. This means you can be online with more then one client 	at the same time.
Setting the jid is a very simple thing: A third argument has to be passed to the auth() method.
So the authentication step may look like this:



\begin{verbatim}
cl.auth(jid.getNode(),pwd,"laptop")
\end{verbatim}


\chapter{Handling presence events}

\section{Simple presence handling}

Presence events are those messages which contain informations about your status and subscription.
You may have wondered why our bot from example 2 wasn't shown as "online" in your contact list.
This was because we didn't notify the server that the bot is online. This could be done by "sendInitPresence()". Go back to example 2 and uncomment the appropriate line. If the bot is in your roster, he will be marked as "online".
\newline
\newline
The next thing about presence covers subscription. "subscription" in generally means: "Allow somebody to see if you're online and allow him to add you to his roster".
To handle this events, we have to register a presence handler. This is done on the same way as the message handler in example 2.

\begin{verbatim}
1  #!/usr/bin/python
2  import sys
3  import xmpp
4  import os
5  import signal
6  import time
7
8  def presenceCB(conn,msg):
9	print str(msg)
10	prs_type=msg.getType()
11	who=msg.getFrom()
12	if prs_type == "subscribe":
13		conn.send(xmpp.Presence(to=who, typ = 'subscribed'))
14		conn.send(xmpp.Presence(to=who, typ = 'subscribe'))
15
16
17  def StepOn(conn):
18    try:
19        conn.Process(1)
20    except KeyboardInterrupt:
21	    return 0
22    return 1
23
24 def GoOn(conn):
25    while StepOn(conn):
26	    pass
27
28
29 def main():
30	jid="user@domain.tld"
31	pwd="secret"
32
33	jid=xmpp.protocol.JID(jid)
34
35	cl = xmpp.Client(jid.getDomain(), debug=[])
36
37	cl.connect()
38
39	cl.auth(jid.getNode(),pwd)
40
41
42	cl.RegisterHandler('presence', presenceCB)
43	cl.sendInitPresence()
44
45	GoOn(cl)
46
47 main()
\end{verbatim}
This is the most simple presence handler. When you receive a message containing "subscribe", return a "subscribed" answer and ask him for subscription. That means subscribing everyone who asks.
You can imagine that this is not the right thing for real world applications. I prefer limits like a maximal roster size and a policy to add users to the roster. This may differ for your bot..
\newline
Think about who should be able to use your bot and block everyone else.
\newpage
\section{Retrieving the status}


Many applications need the status of an user. They want to know if he's online or offline or maybe only away.
This is not as easy as it seems. There's no direct way to get the Status of a given jid. You have to analyse the presence messages the oponnents send. If a user gets online or if you go online, everyone who has subscribed to you will send a presence message. Of course this works only if the user is online.
If a user logs out, he sends you a presence message with the Status "Logged out". Everytime he changes his status, you will receive a corresponding status message.
My workaround for the problem: keep track of the actual status with a dictionary. Take the jid as the key
and set their value if you receive a presence message for the jid.
\newline
\newline
If you know a better way to do this, please contact me !
\section{Roster management}


The roster is the jabber synonym for the better known term "contact list". You can easily fetch, add or remove entries.
In the most cases you won't have to deal with this, because the most actions are done automatically.
For example if you subscribe to someones presence, he will be added to your roster.
Therefore xmpppy offers you a higher level class to operate on your roster: x.roster.Roster
The functions should be self-explanatory if you have a look at
\newline \textit{http://xmpppy.sourceforge.net/apidocs/public/x.roster.Roster-class.html}.


\chapter{Advanced concepts}
%groupchat.tex
\section{Groupchat}

I suppose most of you have experiences with many-to-many chats like IRC. Jabber has its own many-to-many chat which is implement by two protocols: \textit{groupchat 1.0} and \textit{multi user chat (MUC)}.
The MUC protocol is based on groupchat and allows a lot of advanced features like user privileges and passwort-protected rooms.
Because of the simple implementation groupchat will focused here.
If you need \textit{MUC}, have a look at \textit{http://www.jabber.org/jeps/jep-0045.html}.
\newline
\newline
Using \textit{groupchat 1.0} is very easy because it is based on presence messages.
First we have to think about what we want to do. Let's imagine that we want to join the room "castle\_anthrax" with the nickname "zoot". The server is called "conference.holy-gra.il".
To enter a room, the only thing to do is sending a presence message with the room and your nickname to your conference server.
\newline
To speak in python:
\begin{verbatim}
room = "castle_anthrax@conference.holy-gra.il/zoot"
cl.send(xmpp.Presence(to=room))
\end{verbatim}
Of course it could happen that someone in this room has already chosen the nickname "zoot". This will cause an error 409 ("Conflict") and we have to try another nickname.
\newline
If nobody named "zoot" is in there, we'll receive a presence message from everybody in that room (including yourself).
The "from" attribut of the message looks like this: \textit{castle\_anthrax@conference.holy-gra.il/galahad}, which means that somebody named galahad is already in this room.This might be interesting for you if you write a jabber client and you have to keep track of the nicks in the room.
\newline
Receiving messages is divided in 2 parts. If someone sends a public message, you'll receive a message with the type "groupchat".This message will be sent to every user in that room.A private message is send as a chat type message.
\newline
Sending message is as easy as receiving messages. Just send a groupchat message to the room-jid or send a chat message to a room-member, if you like some private chit-chat..
\newline
\newline
The following example is based on recv.py, the script used in chapter 2: receiving messages.


\begin{verbatim}
#!/usr/bin/python
import sys
import xmpp
import os
import signal
import time

def messageCB(conn,msg):
    if msg.getType() == "groupchat":
         print str(msg.getFrom()) +": "+  str(msg.getBody())
    if msg.getType() == "chat":
         print "private: " + str(msg.getFrom()) +  ":" +str(msg.getBody())

def presenceCB(conn,msg):
    print msg




def StepOn(conn):
    try:
         conn.Process(1)
    except KeyboardInterrupt:
         return 0
    return 1

def GoOn(conn):
    while StepOn(conn):
         pass


def main():

    jid="user@domain.tld"
    pwd="secret"

    jid=xmpp.protocol.JID(jid)

    cl = xmpp.Client(jid.getDomain(), debug=[])

    cl.connect()

    cl.auth(jid.getNode(),pwd)


    cl.sendInitPresence()

    cl.RegisterHandler('message', messageCB)

    room = "castle_anthrax@conf.holy-gra.il/zoot"
    print "Joining " + room

    cl.send(xmpp.Presence(to=room))
    cl.send(xmpp.Message(castle_anthrax@conf.holy-gra.il,"zoot is here!","groupchat"))
    cl.send(xmpp.Message(castle_anthrax@conf.holy-gra.il/galahad,"hey sweetheart","chat"))

    GoOn(cl)

main()
\end{verbatim}
\chapter{Something about Knigge}

If you don't know Knigge, see \textit{http://en.wikipedia.org/wiki/Knigge}.
\newline
His main work is a book called "Ueber den Umgang mit Menschen" ("On Human Relations").
\newline
As it appears clear that there are (social) rules for human relation, not everybody knows that there a rules
for bot communication too.
\newline
\newline
As i already noted in chapter 3, developing bots is more than a technical thing.
You should consider that you are responsible for your bot. Imagine you got something wrong and your bot is crashing his server by sending messages every millisecond. That's not a great deal if you use a dedicated server for testing, but that's unusual.
\newline
\newline
The following hints are loosely based on the rules given by
\newline
\textit{http://web.swissjabber.ch\/index.php/Regeln\_zum\_Betrieb\_von\_Bots\_und\_Robots}.
As a result of abuse by out-of-control bots, some server admins ban every bot which is not written with the following rules in mind:
\begin{itemize}
\item use a "help" command which identifies the function and the administrator of your bot
\item ignore messages from jids not including a "@"
\item perfom regular logins only in an interval of five minutes
\item reply only to jids with the status: available, chat, away, xa or dnd
\item don't reply on error messages
\end{itemize}



%ressources.tex
\chapter{Ressources}


\section{Websites}
xmpppy homepage:		\textit{http://xmpppy.sourceforge.net}
\newline
Jabber foundation:		\textit{http://jabber.org}
\newline
\newline
RFC 3920: xmpp core		\textit{http://www.ietf.org/rfc/rfc3920.txt}
\newline
RFC 3921: IM and presence	\textit{http://www.ietf.org/rfc/rfc3921.txt}
\newline
RFC 3922: CPIM			\textit{http://www.ietf.org/rfc/rfc3922.txt}
\newline
RFC 3923: End2End signing and Object Encryption	\textit{http://www.ietf.org/rfc/rfc3923.txt}
\newline
\newline
Python central:			\textit{http://python.org}
\newline
psi client:			\textit{http://psi.affinix.com/}

\section{Books}
Adams,DJ: Programming jabber	\textit{http://www.oreilly.com/catalog/jabber/}

\section{Projects using xmpppy}
Gajim			\textit{http://www.gajim.org/}


\newpage
\end{document}

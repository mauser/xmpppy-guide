%knigge.tex
\section{Something about Knigge}

If you don't know Knigge, see \textit{http://en.wikipedia.org/wiki/Knigge}.
\newline
His main work is a book called "Ueber den Umgang mit Menschen" ("On Human Relations").
\newline
As it appears clear that there are (social) rules for human relation, not everybody knows that there a rules
for bot communication too.
\newline
\newline
As i already noted in chapter 3, developing bots is more than a technical thing.
You should consider that you are responsible for your bot. Imagine you got something wrong and your bot is crashing his server by sending messages every millisecond. That's not a great deal if you use a dedicated server for testing, but that's unusual.
\newline
\newline
The following hints are loosely based on the rules given by
\newline
\textit{http://web.swissjabber.ch\/index.php/Regeln\_zum\_Betrieb\_von\_Bots\_und\_Robots}.
As a result of abuse by out-of-control bots, some server admins ban every bot which is not written with the following rules in mind:
\begin{itemize}
\item use a "help" command which identifies the function and the administrator of your bot
\item ignore messages from jids not including a "@"
\item perfom regular logins only in an interval of five minutes
\item reply only to jids with the status: available, chat, away, xa or dnd
\item don't reply on error messages
\end{itemize}



\chapter{Prerequisites}

Before we start with coding, you should check some preconditions. An important thing is the right jabber client. I would therefore recommend the use of psi. Psi is a feature-rich jabber-only client. It supports service browsing and it implements the most features as described in the RFC. Especially the group chat support is better implemented than in the most other clients.
\newline
Of course you can use a mutliprotocol client like Gaim or kopete too, but they implement some things on a different way.. For example gaim makes no difference between the multiple message types (chat, message..).
\newline
\newline
If you want to develop something with groupchat, you may want to setup your own jabber server.
This may be recommended if you're new to xmpppy and you don't know exactly what you're doing.
Think about the possibility of disturbing other people or - in the worst case - killing a server.
If your bot gets out of control and sends every 10ms a message to the server, the server admin might be a little bit disgusted of your public beta testing. So keep in mind that there are other people (and of course bots) out there while your bot uses a public server.
\newline
\newline
Furthermore i assume that you have already registered a jabber account.
\newline
\newline
If you're completely new to jabber, you should read something about the underlying techniques such as XML-Streams. A good starting point is the xmpp-core RFC.
